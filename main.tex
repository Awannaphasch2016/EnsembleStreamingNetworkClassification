% Created 2022-03-07 Mon 12:47
% Intended LaTeX compiler: pdflatex
\documentclass{IEEEtran}
\usepackage[utf8]{inputenc}
\usepackage[T1]{fontenc}
\usepackage{graphicx}
\usepackage{longtable}
\usepackage{wrapfig}
\usepackage{rotating}
\usepackage[normalem]{ulem}
\usepackage{amsmath}
\usepackage{amssymb}
\usepackage{capt-of}
\usepackage{hyperref}
\author{Anak Wannaphaschaiyong}
\date{\textit{<2022-03-03 Thu>}}
\title{Ensemble Approaches for Streaming Networking Classification}
\begin{document}

\maketitle

\section{Introduction}
\label{sec:org689b55d}
Dunamic graph is vaguely used term. In general, dynamic graph is an ordered list of node and link events. These events include deletion and addition of nodes and edges after a interval of time.
Dynamic graph is known in other named as streaming graph and temporal graph. Concretely, dynamic graph can be categorized into subcategories \cite{barrosSurveyEmbeddingDynamic2021,kazemiRepresentationLearningDynamica,skardingFoundationsModelingDynamic2021}

\section{Related Work}
\label{sec:org578a47b}
\subsection{Sliding Window Evaluation}
\label{sec:org9b01a27}
\subsection{Dynamic Graph}
\label{sec:orgb544e98}

\bibliographystyle{plain}
\bibliography{../../../org/papers/zotero-bib}
\end{document}
